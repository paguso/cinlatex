% Use opções compatívels com a classe article
\documentclass[12pt, oneside]{ufpeexam}
% Use a opção bw para tipografar em preto e branco
%\documentclass[12pt, oneside, bw]{ufpeexam}

\usepackage{multicol}
\usepackage{paralist}
\usepackage{lipsum}

%% Caso deseje, pode mudar os rótulos-padrão das questões e suas soluções
%\makeatletter
%\renewcommand{\@question}{Pergunta}
%\renewcommand{\@answer}{Resposta}
%\makeatother


% Identificação da Universidade / Centro / Depto
\university{Universidade Federal de Pernambuco}
\institute{Centro de Informática}
%\department{Depto. de Ciência da Computação}

% Identificação do curso
\program{Nome do Curso}

% Identificação da disciplina
\course{Nome da Disciplina}

% Identificação do semestre letivo
\term{2020.2}

% Professor
\author[a]{Nome da Professora} % author[a]=female author[s]=multiple

% Descrição do Exame
\title{Primeira Prova}


\date{\today}

%% Descomente a linha a seguir para omitir as soluções
%\hideanswers

\begin{document}

\makeheader

%% Insere uma linha para assinatura do aluno, se necessário
% \studentsignature

%% Adicione uma caixa de instruções com o ambiente instructions
\begin{instructions}
\begin{compactitem}
\item Esta prova tem XXX questões.
\item A duração da prova é de 02h00min.
\end{compactitem}
\end{instructions}

%% Se preferir, pode tipografar a prova em duas colunas, para poupar espaço
%\begin{multicols}{2}

%% Adicione questões com o ambiente question
%% O argumento opcional é inserido logo após o rótulo da questão
\begin{question}[\smallskip (2,0pt)\\]
%% Texto da questão aqui
	\lipsum[1-2][3-4]
\end{question}

%% Adicione uma resposta com o ambiente answer
%% Para omitir a solução, use o comando \hideanswers antes do \begin{document}
%% Assim você pode preparar as soluções mas só exibi-las posteriormente na
%% divulgação do gabarito.
\begin{answer}
	\lipsum[5-6][7-8]
\end{answer}


\begin{question}[\smallskip (2,0pt)\\]
	\lipsum[9-11][12-15]
\end{question}
\begin{answer}
	\lipsum[16-19][20-23]
\end{answer}


\begin{question}[\smallskip(\textit{Uma questão muito longa}: 8,0pt)\\]
	\lipsum[24-30][1-10]
\end{question}
\begin{answer}
	\lipsum[30-40][10-60]
\end{answer}



%% Fim do trecho em duas colunas
%\end{multicols}

%% Apenas marca o fim da prova
\eof

\end{document}
